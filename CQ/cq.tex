\documentclass{article}

\usepackage{amsmath, amssymb}
\usepackage[utf8]{inputenc}

\newcommand{\fou}[1]{\ensuremath{
	\mathcal{F}\left\{#1\right\}
}}

\newcommand{\ifou}[1]{\ensuremath{
		\mathcal{F}^{-1}\left\{#1\right\}
}}

\newcommand{\fil}[1]{\ensuremath{
	\mathbf{T}\left[#1\right]
}}

\newcommand{\infsum}[1]{\ensuremath{
	\sum_{#1=-\infty}^{\infty}
}}

\newcommand{\infint}{\ensuremath{
		\int_{-\infty}^{\infty}
}}

\newcommand{\vect}[3]{\ensuremath{
	\left[\begin{matrix}
		#1\\#2\\#3
	\end{matrix}\right]
}}

\newcommand{\vectt}[3]{\ensuremath{
		\left[\begin{matrix}
			#1& #2& #3
		\end{matrix}\right]^T
}}

\newcommand{\expi}[2][]{\ensuremath{
		e^{#12 \pi j #2}
}}


\title{Telecom 2019.2 - P2$^\dagger$}
\begin{document}
	\section*{Questão 1. {\small Filtro Linear Invariante no Tempo (L.T.I.)}}
	Se $\fil{\cdot}$ é um filtro linear, então vale que
		$$\fil{\alpha \mathbf{x}(t) + \beta \mathbf{y}(t)} = \alpha \fil{\mathbf{x}(t)} + \beta \fil{\mathbf{y}(t)}$$
	Se $\fil{\cdot}$ é invariante no tempo e se $\mathbf{y}(t) = \fil{\mathbf{x}(t)}$ então
		$$\fil{\mathbf{x}(t + \Delta t)} = \mathbf{y}(t + \Delta t)$$
	Sabido isso, podemos concluir que $\fil{\cdot}$ pode ser expresso matematicamente por uma convolução no domínio do tempo entre o sinal de entrada e a resposta ao impulso $\mathbf{h}(t)$, ou seja,
		$$\mathbf{h}(t) = \fil{\mathbf{\delta}(t)}$$
	onde
		$$\mathbf{\delta}(t) = 
		\begin{cases}
			\displaystyle \int_{-\infty}^{\infty} \mathbf{\delta}(t) = 1\\
			0, t \neq 0
		\end{cases}$$
	Como ele é L.T.I, temos também que
		$$\mathbf{h}(t + \tau) =  \fil{\mathbf{\delta}(t + \tau)} \ \ \forall \tau$$
	Supomos que 
		$$x(n) = x(n) \infsum{i} \mathbf{x}(i) \cdot \mathbf{\delta}(i - n)$$
	Dessa forma
		\begin{align*}
			\fil{x(n)}  &= \fil{\infsum{i} \mathbf{x}(i) \cdot \mathbf{\delta}(i - n)}\\
						&= \infsum{i} \fil{\mathbf{x}(i) \cdot \mathbf{\delta}(i - n)}\\
						&= \infsum{i} \mathbf{x}(i) \cdot \fil{\mathbf{\delta}(i - n)}\\
						&= \infsum{i} \mathbf{x}(i) \cdot \mathbf{h}(i - n)
		\end{align*}
		
	\section*{Questão 2. {\small Cálculo do centróide}}
	\textbf{Enunciado.} Dados os vetores $\vect{1}{2}{3}$, $\vect{3}{1}{1}$ e $\vect{1}{2}{1}$ calcule o centróide.
	~\\
	~\\
	~\\
	\textbf{Resposta.} Sejam $\vec{\mathbf{x}}_1 = \vect{1}{2}{3}$, $\vec{\mathbf{x}}_2 = \vect{3}{1}{1}$ e $\vec{\mathbf{x}}_3 = \vect{1}{2}{1}$.
	~\\
	~\\
	~\\
	Centróide é o ponto com a menor distância média em relação aos demais pontos da amostra. O termo $\frac{1}{N}$ não é importante e podemos usar a distância ao quadrado para comparação. Portanto, calculamos para cada $\vec{\mathbf{x}}_i$
		$$d_i^2 = \sum_{i \neq j} ||\vec{\mathbf{x}}_i - \vec{\mathbf{x}}_j||^2$$
	
	\begin{align*}
	d_1^2 &= ||\vec{\mathbf{x}}_1 - \vec{\mathbf{x}}_2||^2 + ||\vec{\mathbf{x}}_1 - \vec{\mathbf{x}}_3||^2\\
	&= \left[(-2)^2 + 1^2 + 2^2\right] + \left[0^2 + 0^2 + 2^2\right]\\
	&= 13\\
	~\\
	d_2^2 &= ||\vec{\mathbf{x}}_2 - \vec{\mathbf{x}}_1||^2 + ||\vec{\mathbf{x}}_2 - \vec{\mathbf{x}}_3||^2\\
	&= \left[2^2 + (-1)^2 + (-2)^2\right] + \left[2^2 + (-1)^2 + 0^2\right]\\
	&= 14\\
	~\\
	d_3^2 &= ||\vec{\mathbf{x}}_3 - \vec{\mathbf{x}}_1||^2 + ||\vec{\mathbf{x}}_3 - \vec{\mathbf{x}}_2||^2\\
	&= \left[0^2 + 0^2 + (-2)^2\right] + \left[(-2)^2 + 1^2 + 0^2\right]\\
	&= 9
	\end{align*}
	
	Como $d_3^2 < d_1^2 < d_2^2$, dizemos que $\vec{\mathbf{x}}_3 = \vect{1}{2}{1}$ é o centróide.
	
	\section*{Questão 3. {\small Modulação AM/FM}}
	
	\section*{Questão Extra. {\small Convolução das Transformadas}}
	\textbf{Enunciado.} Demonstre as seguintes identidades:
	\begin{align*}
		&\text{a) }\fou{a(t) \cdot b(t)} = \fou{a(t)} \ast \fou{b(t)}\\
		&\text{b) }\fou{a(t) \ast b(t)} = \fou{a(t)} \cdot \fou{b(t)}
	\end{align*}
	
	\textbf{Resposta.}
	
	\begin{itemize}
		\item [a)]
		Pela definição
			$$\fou{a(t) \cdot b(t)} = \infint a(t) \cdot b(t) \cdot \expi[-]{ft}\ dt$$
		No entanto,
			$$b(t) = \ifou{B(f)} = \infint B(f) \cdot \expi{f t} df$$
		e portanto,
			\begin{align*}
			\fou{a(t) \cdot b(t)} &= \infint a(t) \cdot \left(\infint B(\hat{f}) \cdot \expi{\hat{f} t} d\hat{f}\right) \cdot \expi[-]{ft}\ dt\\
			&= \infint B(\hat{f}) \infint a(t) \cdot \expi[-]{(f - \hat{f}) t}\ dt\ d\hat{f}\\
			&= \infint B(\hat{f}) \cdot A(f - \hat{f}) d\hat{f}\\
			\fou{a(t) \cdot b(t)} &= B(f) \ast A(f) = A(f) \ast B(f) = \fou{a(t)} \ast \fou{b(t)}\\
			&& \blacksquare
			\end{align*}
		\item [b)]
			Seguindo o mesmo raciocínio,
			$$\fou{a(t) \ast b(t)} = \infint \left(\infint a(\tau) \cdot b(t - \tau)\ d\tau\right) \expi[-]{f t}\ dt$$
	\end{itemize}
\end{document}