\documentclass[14pt]{beamer}
% \usetheme{metropolis}
% \usetheme{Warsaw}
\usepackage[utf8]{inputenc}

\usepackage[absolute,overlay]{textpos}
\usepackage{graphicx}

\usepackage{courier}


\title{\texttt{Computação Quântica}}
\institute{IM-UFRJ}
\date{19 de novembro de 2019}

\author{Pedro Maciel Xavier}


\begin{document}

%	\frame{\titlepage}
	

	\section{Computação Digital}	
	\frame%
	{
		\title{Transistor}
	}

	\frame%
	{
		\title{Portas Lógicas}
	}

	\frame%
	{
		\frametitle{Arquitetura de Von Neuman}
		
		\begin{textblock*}{4cm}(2cm, 2cm) % {block width} (coords)
			\includegraphics[width=4cm]{von-neuman.jpg}
			John Von Neuman\\
			{\centering\small 1903 - 1957}
		\end{textblock*}
	}

	\section{Historinha}
	\frame%
	{
		\frametitle{Fenômenos Quânticos}
		
		\begin{textblock*}{4cm}(2cm, 2cm) % {block width} (coords)
			\includegraphics[width=4cm]{feynman.jpg}
			Richard Feynman\\
			{\centering\small 1918 - 1988}
		\end{textblock*}		
	}
	
	\frame%
	{
				
	}

	\frame%
	{
				
	}
	
	\section{Algoritmos Quânticos}
	\frame%
	{
		\frametitle{Algoritmo de Shor}
		
		\begin{textblock*}{4cm}(2cm, 2.5cm) % {block width} (coords)
			\includegraphics[width=4cm]{shor.jpg}
			Peter Shor\\
			{\centering\small MIT}
		\end{textblock*}		
	}

	\frame%
	{
		\frametitle{Algoritmo de Grover}
		
		\begin{textblock*}{4cm}(2cm, 2.5cm) % {block width} (coords)
			\includegraphics[width=4cm]{grover.jpg}
			Lov Grover\\
			{\centering\small Bell Labs}
		\end{textblock*}		
	}
	
	\frame%
	{
				
	}
	
	\frame%
	{
				
	}

	\frame%
	{
				
	}
	
	\frame%
	{
				
	}
	
	\frame%
	{
				
	}
	
	\frame%
	{
				
	}
\end{document}